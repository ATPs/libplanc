% !TEX root = experiments.tex

% macros for plotting
\newcommand{\datafile}{}
\newcommand{\alg}{}
\newcommand{\numiterations}{30}
\newcommand{\minvalue}{1}

% toggles whether or not to plot naive results (if not, the rows of the data file also need to be commented out)
\newif\ifnaive
% toggles ksweep vs scaling plot
\newif\ifksweep
% toggles legend
\newif\iflegend
% toggles y axis label
\newif\ifylabel
% toggles wider plot area
\newif\ifwider
% toggles including liavas results
\newif\ifliavas


\newcommand{\setcolors}{
\pgfplotsset{cycle list={
	red, fill=red \\ 
	blue, fill=blue \\ 
	green, fill=green \\ 
	red, pattern=crosshatch, pattern color=red \\
	blue, pattern=crosshatch, pattern color=blue \\
	green, pattern=crosshatch, pattern color=green \\
}};
}

% set options for grouped bar plot
\newcommand{\breakdownplotoptions}{
	ybar stacked,
	reverse legend,
	bar width=8pt,
	width=9cm, height=3.85cm,
	ylabel={Time (s)}, 
	%y label style={yshift=-.5cm},
	ymin=0,
	symbolic x coords={D1,N1,,D16,N16,,D81,N81,,D256},
	xtick=data,
	legend style={at={(0.5,1.3)},anchor=north},
	legend columns=-1,
	reverse legend
}

% stacked bar plot command
\newcommand{\breakdownplot}{
\begin{axis}[\breakdownplotoptions]
	\setcolors
	\addplot table[x=alg-p, y expr=(\thisrow{mttkrp}/(\minvalue*\numiterations))] {\datafile};
	\addplot table[x=alg-p, y expr=(\thisrow{krp}/(\minvalue*\numiterations))] {\datafile};
	\addplot table[x=alg-p, y expr=((\thisrow{allgather}+\thisrow{reducescatter})/(\minvalue*\numiterations))] {\datafile};
	\addplot table[x=alg-p, y expr=((\thisrow{nnls}+\thisrow{gram}+\thisrow{allreduce})/(\minvalue*\numiterations))] {\datafile};
	\addplot table[x=alg-p, y expr=((\thisrow{err_compute}+\thisrow{err_communication})/(\minvalue*\numiterations))] {\datafile};
	\legend{MTTRKP,KRP,Factor Comm,NLS,Error};
\end{axis}
}

% labels for bar groups (manually positioned)
\newcommand{\labels}{
%\node [align=center,text width=3cm] at (1.25cm, -.95cm)   {\ifksweep 10 \else 16 \fi};
%\node [align=center,text width=3cm] at (3cm, -0.95cm)   {\ifksweep 20 \else 96 \fi};
%\node [align=center,text width=3cm] at (4.6cm, -0.95cm)   {\ifksweep 30 \else 384 \fi};
%\node [align=center,text width=3cm] at (6.3cm, -0.95cm) {\ifksweep 40 \else 864 \fi};
%\node [align=center,text width=3cm] at (8.15cm, -0.95cm) {\ifksweep 50 \else 1536 \fi};
\node [align=center,text width=3cm] at (1cm, -1.15cm)   {\ifksweep 10 \else 1 \fi};
\node [align=center,text width=3cm] at (2.4cm, -1.15cm)   {\ifksweep 20 \else 16 \fi};
\node [align=center,text width=3cm] at (3.7cm, -1.15cm)   {\ifksweep 30 \else 81 \fi};
\node [align=center,text width=3cm] at (5.1cm, -1.15cm) {\ifksweep 40 \else 256 \fi};
}

% allows for filtering rows from dat file
\pgfplotsset{
    discard if not/.style 2 args={
        x filter/.append code={
            \edef\tempa{\thisrow{#1}}
            \edef\tempb{#2}
            \ifx\tempa\tempb
            \else
                \def\pgfmathresult{inf}
            \fi
        }
    }
}

% set options for strongscaling plot
\newcommand{\strongscalingplotoptions}{
	log basis y={2},
	log basis x={2},
	xlabel=Nodes,
	ylabel=Time (s),
	y tick label style={
        /pgf/number format/.cd,
            precision=4,
	},
	legend style={draw=none, cells={align=left}, nodes={scale=0.7}}
}

% plot time over p, filtering rows out appropriately
\newcommand{\strongscalingplot}{
\begin{loglogaxis}[\strongscalingplotoptions]
	%\addplot+ [discard if not={alg}{N},thick,mark options={solid},mark=square*] table [x={p}, y={totaltime}] {\datafile};
	\addplot+ [discard if not={alg}{D},thick,mark options={solid},mark=triangle*] table [x={p}, y expr=(\thisrow{total}/(\minvalue*\numiterations))] {\datafile};
	\addplot+ [discard if not={alg}{N},thick,mark options={solid},mark=square*] table [x={p}, y expr=(\thisrow{total}/(\minvalue*\numiterations))] {\datafile};
	\ifliavas
		\addplot+ [discard if not={alg}{L},thick,mark options={solid}] table [x={p}, y expr=(\thisrow{total}/(\minvalue*\numiterations))] {\datafile};
		\legend{DimTree,NoDimTree,Liavas}
	\else
		\legend{DimTree,NoDimTree};
	\fi
\end{loglogaxis}
}

