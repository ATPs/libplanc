% !TEX root = paper.tex

\section{Performance Results} \label{sec:experiments}

% !TEX root = experiments.tex

% macros for plotting
\newcommand{\datafile}{}
\newcommand{\alg}{}
\newcommand{\numiterations}{30}
\newcommand{\minvalue}{1}

% toggles whether or not to plot naive results (if not, the rows of the data file also need to be commented out)
\newif\ifnaive
% toggles ksweep vs scaling plot
\newif\ifksweep
% toggles legend
\newif\iflegend
% toggles y axis label
\newif\ifylabel
% toggles wider plot area
\newif\ifwider
% toggles including liavas results
\newif\ifliavas


\newcommand{\setcolors}{
\pgfplotsset{cycle list={
	red, fill=red \\ 
	blue, fill=blue \\ 
	green, fill=green \\ 
	red, pattern=crosshatch, pattern color=red \\
	green, pattern=crosshatch, pattern color=green \\
	blue, pattern=crosshatch, pattern color=blue \\
}};
}

% set options for grouped bar plot
\newcommand{\breakdownplotoptions}{
	ybar stacked,
	reverse legend,
	bar width=8pt,
	width=9cm, height=3.85cm,
	ylabel={Time (s)}, 
	y label style={yshift=-.5cm},
	ymin=0,
	symbolic x coords={D1,N1,D16,N16,D81,N81,D256,N256},
	xtick=data,
	legend style={at={(0.5,1.3)},anchor=north},
	legend columns=-1,
	reverse legend
}

% stacked bar plot command
\newcommand{\breakdownplot}{
\begin{axis}[\breakdownplotoptions]
	\setcolors
	\addplot table[x=alg-p, y expr=(\thisrow{mttkrp}/(\minvalue*\numiterations))] {\datafile};
	\addplot table[x=alg-p, y expr=(\thisrow{krp}/(\minvalue*\numiterations))] {\datafile};
	\addplot table[x=alg-p, y expr=((\thisrow{allgather}+\thisrow{reducescatter})/(\minvalue*\numiterations))] {\datafile};
	\addplot table[x=alg-p, y expr=((\thisrow{nnls}+\thisrow{gram}+\thisrow{allreduce})/(\minvalue*\numiterations))] {\datafile};
	\addplot table[x=alg-p, y expr=((\thisrow{err_compute}+\thisrow{err_communication})/(\minvalue*\numiterations))] {\datafile};
	\legend{MTTKRP,KRP,Factor Comm,NLS,Error};
\end{axis}
}

% labels for bar groups (manually positioned)
\newcommand{\labels}{
%\node [align=center,text width=3cm] at (1.25cm, -.95cm)   {\ifksweep 10 \else 16 \fi};
%\node [align=center,text width=3cm] at (3cm, -0.95cm)   {\ifksweep 20 \else 96 \fi};
%\node [align=center,text width=3cm] at (4.6cm, -0.95cm)   {\ifksweep 30 \else 384 \fi};
%\node [align=center,text width=3cm] at (6.3cm, -0.95cm) {\ifksweep 40 \else 864 \fi};
%\node [align=center,text width=3cm] at (8.15cm, -0.95cm) {\ifksweep 50 \else 1536 \fi};
\node [align=center,text width=3cm] at (1cm, -1.15cm)   {\ifksweep 10 \else 1 \fi};
\node [align=center,text width=3cm] at (2.4cm, -1.15cm)   {\ifksweep 20 \else 16 \fi};
\node [align=center,text width=3cm] at (3.7cm, -1.15cm)   {\ifksweep 30 \else 81 \fi};
\node [align=center,text width=3cm] at (5.1cm, -1.15cm) {\ifksweep 40 \else 256 \fi};
}

% allows for filtering rows from dat file
\pgfplotsset{
    discard if not/.style 2 args={
        x filter/.append code={
            \edef\tempa{\thisrow{#1}}
            \edef\tempb{#2}
            \ifx\tempa\tempb
            \else
                \def\pgfmathresult{inf}
            \fi
        }
    }
}

% set options for strongscaling plot
\newcommand{\strongscalingplotoptions}{
		log basis y={2},
		log basis x={2},
		\ifliavas
			xlabel=Cores,
		\else
			xlabel=Nodes,
		\fi
		ylabel=Time (s),
		y tick label style={
	        /pgf/number format/.cd,
            	precision=4,
		},
	legend style={draw=none, cells={align=left}, nodes={scale=0.7}}
}

% plot time over p, filtering rows out appropriately
\newcommand{\strongscalingplot}{
\begin{loglogaxis}[\strongscalingplotoptions]
	%\addplot+ [discard if not={alg}{N},thick,mark options={solid},mark=square*] table [x={p}, y={totaltime}] {\datafile};
	\ifliavas
		\addplot+ [discard if not={alg}{DF},thick,mark options={solid},mark=square*] table [x={p}, y expr=(\thisrow{total}/(\minvalue*\numiterations))] {\datafile};
		\addplot+ [discard if not={alg}{NF},thick,mark options={solid},mark=square*] table [x={p}, y expr=(\thisrow{total}/(\minvalue*\numiterations))] {\datafile};
	\else
		\addplot+ [discard if not={alg}{D},thick,mark options={solid},mark=triangle*] table [x={p}, y expr=(\thisrow{total}/(\minvalue*\numiterations))] {\datafile};
		\addplot+ [discard if not={alg}{N},thick,mark options={solid},mark=square*] table [x={p}, y expr=(\thisrow{total}/(\minvalue*\numiterations))] {\datafile};
	\fi
	\ifliavas
		\addplot+ [discard if not={alg}{L},thick,mark options={solid}] table [x={p}, y expr=(\thisrow{total}/(\minvalue*\numiterations))] {\datafile};
		\legend{DimTree,NoDimTree,NbAO-NTF \cite{LK+17b}}
	\else
		\legend{DimTree,NoDimTree,FlatDimTree,FlatNoDimTree};
	\fi
\end{loglogaxis}
}



\subsection{Datasets}

\subsubsection{Hyperspectral Images (HSI)}
For comparison with previous work \cite{LK+17b}, we consider the same 3D hyperspectral imaging dataset called ``Souto\_wood\_pile'' \cite{FAN16}. 
NNCP is often used on HSI data sets for classification and blind source separation of materials with differing spectral signatures.
The hyperspectral datacube has dimensions $1024 \times 1344 \times 33$ and represents a set of 33 grayscale images of size 1344 $\times$ 1024 pixels sampled at wavelengths 400, 410, $\dots$, 720 nm, with each pixel value representing spectral radiance in $W m^{-2} sr^{-1} nm^{-1}$. 
We also consider the Nogueir\'{o} scene dataset, which is a sequence of 9 time-lapse HSI images of the same scene acquired at about 1-hour intervals.
In each scene, hyperspectral images were acquired at about 1-hour intervals. 
Each Nogueir\'{o} scene HSI image has the same properties as the Souto\_wood\_pile data set, so the corresponding tensor has dimensions $1024 \times 1344 \times 33 \times 9$. 

\subsubsection{Dynamic Functional Connectivity (dFC)}
We also consider dynamic functional connectivity datasets that are generated from fMRI images of human brains.
Given a 4D fMRI data set of voxel measurements across multiple timesteps, voxels containing brain data are partitioned into a set of regions of interest (specified using domain-specific knowledge), and a single time-series signal is aggregated for each region of interest.
Then, an instantaneous correlation is computed for each time point and pair of regions, and this process is repeated for a number of subjects.
Computing a CP decomposition of this data helps to discover patterns of brain connectivity among different regions and also differentiate among individuals.
For our representative dFC data set, we consider 246 brain regions, which yields 30{,}012 unique pairs of regions, 1200 times steps, and 500 subjects, or a tensor of dimension $30{,}012\times 1200 \times 500$ \cite{VEU+12,THBGW17}.

\subsubsection{Synthetic}
Our synthetic data sets are constructed from a CP model with an exact low rank with no added noise.
In this case we can confirm that the residual error of our algorithm with a random start converges to zero.
For the purposes of benchmarking, we run a fixed number of iterations of the BCD algorithm rather than using a convergence check.

%For the synthetic datasets, our open source code supports (a) low-rank tensor (b) uniform random and (c) positive shifted normal distribution of $\mathfrak{N}(3,1)$ -- that is change the mean such that all the random numbers are positive. We considered three different synthetic matrices for different cases. For baseline comparison with Liavas \cite{LK+17b}, we considered a three mode 
%uniform of size 1024x1024x1024 on processor grids $2^k \times 2^k \times 2^k$ for $k \in {0,1,2,3}$. We used the uniform five 
%mode synthetic tensor with dimension $64\times 64\times 64\times 64\times 64$ on processor grids 
%$1\times1\times1\times1\times1$, $2\times1\times1\times1\times1$, $\dots$, $2\times2\times2\times2\times2$ for strong scaling 
%experiments.  In the case of weak scaling of four mode synthetic tensors with (D) and without (N) the use of dimension trees.  The 
%tensor and processor grid dimensions are $128k\times 128k\times 128k\times 128k$ and $k\times k\times k\times k$ for $k\in\{1,2,3,4\}$. In all the cases the dimensions were considered such that synthetic tensors can be accommodated even on single node with 64GB for scale up plots. 

\subsection{Machine Details}
The entire experimentation was performed on Eos, a supercomputer at the Oak Ridge Leadership Computing Facility. 
Eos is a 736-node Cray XC30 cluster of Intel Xeon E5-2670 processors with a total of 47.104TB of memory. 
Its compute nodes are organized in blades where each blade contains 4 nodes, and every node has 2 sockets with 8 physical cores and 64GB memory. 
The machine support Intel's hyper-threading (HT), but we restricted it because HT offers minimal improvement for BLAS and LAPACK operations. 
In total, the Eos compute partition contains 11,776 traditional processor cores and our experiments used up to 4,096 cores (35\% of the machine). 

Our objective of the implementation is using open source software as much as possible 
to promote reproducibility and reuse of our code.
%Unlike Liavas \cite{LK+17b} that uses Eigen matrix library \cite{eigenweb}, 
We use Armadillo \cite{sanderson2010} for matrix representation
and operations.  
In Armadillo, the elements of the dense matrix are stored in column major order.
For dense BLAS and LAPACK operations, we linked Armadillo with the default LAPACK/BLAS wrappers from Cray. 
For compiler, we use GNU C++ Compiler (g++ (GCC) 6.3.0) and MPI library is from Cray.  We could also 
compile and run the code in Rhea the commodity cluster from OLCF with entire open source libraries such as OpenBLAS and OpenMPI. 

\subsection{Comparison Implementations}
The implementation proposed by Liavas et al. \cite{LK+17b} is the only publicly available distributed-memory software (of which we are aware) for computing the CP decomposition of dense tensors, with or without constraints.
We use the acronym NbAO-NTF for Nesterov-based Alternating Optimization Nonnegative Tensor Factorization to refer to their code.

It is based on the same parallel algorithm as our implementation, though it is limited to 3D tensors.
The code uses MPI collectives for communication and Eigen \cite{eigenweb} as an interface to BLAS and LAPACK.
We compiled the code linked to BLAS/LAPACK wrappers from Cray  (the same BLAS implementation used by our code) but we were unable to run multithreaded BLAS with their code.
For fair comparison, we use a flat MPI configuration (one MPI process per core) on all comparisons between the two implementations.

We also point out a difference between the Nesterov-based algorithm and the BPP algorithm for solving the NLS subproblems.
The Nesterov-based algorithm attempts an acceleration step using a linear combination of the current and proposed future step; however, it re-computes the residual error before deciding whether or not to accept or reject the acceleration step.
This residual error cannot always be computed cheaply, using the technique described in \cref{sec:error}, and it contributes significantly (approximately 25\%) to the overall run time.
Because the BPP algorithm does not require this extra computation, and studying convergence behavior of the different NLS algorithms is beyond the scope of this work, we remove the time spent in the acceleration step of NbAO-NTF in all our comparisons.

Our proposed algorithm uses dimension trees, but we also benchmark our implementation without that optimization to highlight its importance.
We use an existing implementation to perform the individual MTTKRPs \cite{HBJT18} with this approach.

\subsection{Strong Scaling}

We perform two strong scaling experiments to compare performance with NbAO-NTF.
The experiments use a cubical synthetic tensor and the HSI image used in \cite{LK+17b}, both of which are 3D.

The performance on the cubical synthetic tensor is shown in \cref{fig:strongsynthetic3D}. 
We can observe from the figure that all the three algorithms scale nearly linearly as the problem remains compute bound. 
Recall from \label{sec:analysis} that the computation scales linearly with $1/P$ while the communication scales with $1/P^{1/N}=1/P^{1/3}$. 
As is evident from the figure, the communication cost does not degrade performance even for thousands of cores. 
Our proposed algorithm with dimension trees is 35\% faster than NbAO-NTF at 512 cores (with similar relative difference for other core counts).
This performance improvement is due in large part to the 50\% reduction in arithmetic operations provided by the dimension tree optimization.
There is little difference in performance between our implementation without dimension trees and NbAO-NTF.

\begin{figure}
\begin{tikzpicture}
\renewcommand{\datafile}{data/str_3D_syn.dat}
\renewcommand{\numiterations}{1}
\liavastrue
\strongscalingplot
\end{tikzpicture}
\caption{Strong scaling of 3D synthetic tensor with dimension $1024\times 1024\times 1024$ on processor grids $2^k\times 2^k\times 2^k$ for $k\in\{0,1,2,3,4\}$.  The rank is fixed at 32.}
\label{fig:strongsynthetic3D}
\end{figure}

\Cref{fig:stronghsi3D} shows the strong scaling on the HSI data. 
In this case, our proposed algorithm with dimension trees is over $2\times$ faster than NbAO-NTF, but part of this speedup is due to differences in the NLS update algorithms.
For the low core count, the dimension tree provides a 60\% speedup compared to the MTTKRP time in NbAO-NTF.
At the high core counts for this experiment, the local MTTKRP is no longer the dominating cost.
%The proposed algorithm with classical MTTKRP was performing better on higher number of processors over the baseline. 
%Similarly, at this stage, the SVD used for Nestrov Non-negative Least Squares on Liavas dominates the cost over MTTKRP that makes it even slower than the our proposed algorithm without dimension trees. 

Finally, we ran on 512 nodes, utilizing the 16 threads per node via parallel BLAS and OMP, for a $1024\times1024\times1024$ synthetic tensor and processor grid $8\times8\times8$. Here we observe a $253\times$ speedup which is the result of each node having less than 20 megabytes of data.

\begin{figure}
\begin{tikzpicture}
\renewcommand{\datafile}{data/str_3D_HSI.dat}
\renewcommand{\numiterations}{10}
\liavastrue
\strongscalingplot
\end{tikzpicture}
\caption{Strong scaling of 3D HSI real world data with dimension 1024 x1344 x 33 on processor grids of $k \times k\times 1$ for $k \in {1, 2, 4, 8, 16, 32}$. The rank is fixed at 32.}
\label{fig:stronghsi3D}
\end{figure}
 
In \cref{fig:strongsynthetic5D}, we benchmark performance for a 5D cubical tensor with each dimension set to 64.
Because the tensor is 5D, we can no longer compare against NbAO-NTF.
We see a $13-16\times$ speed up using a dimension tree over not using one for this problem.
As predicted, the dimension tree optimization saves relatively more arithmetic for higher-order tensors.
However, the reduction in leading order flop cost is only $2.5\times$ for $N=5$; the rest of the speedup comes from more efficient DGEMM performance and avoiding memory-bound KRP computation.  
That is, although the flop count of KRP computation is lower order, it still contributes to the run time because it is inefficient.
Also, for tensors with balanced dimensions, the dimension tree approach yields more favorable shapes for DGEMM.


\begin{figure}
\begin{tikzpicture}
\renewcommand{\datafile}{data/str_5D_syn.dat}
\renewcommand{\numiterations}{10}
\liavasfalse
\strongscalingplot
\end{tikzpicture}
\caption{Strong scaling of 5D synthetic tensor with dimension $64\times 64\times 64\times 64\times 64$ on processor grids $1\times1\times1\times1\times1$, $2\times1\times1\times1\times1$, $\dots$, $2\times2\times2\times2\times2$.  The rank is fixed at 32.}
\label{fig:strongsynthetic5D}
\end{figure}

\subsection{Weak Scaling Time Breakdown}

We also perform a weak scaling experiment to understand the time it takes to solve bigger problems with more processors.
In this experiment, we use a synthetic 4D tensor and keep the amount of tensor data assigned to each processor constant, with tensor and processor grid of dimensions $128k\times 128k\times 128k\times 128k$ and $k\times k\times k\times k$ for $k\in\{1,2,3,4\}$ and the rank fixed at 32. 
%For eg., in the case of 3 x 3 x 3 x 3 processor grid with 81 nodes, we ran the proposed algorithm with and without dimension trees on a four mode tensor of size 384 x 384 x 384 x 384. 
The results of the breakdown plot is shown in \cref{fig:weaksynthetic4D}. 
In this case, the algorithm is compute bound with and without the use of the dimension tree, so the total time of the weak scaling remains fixed for both cases. 
Using a dimension tree, the time is completely dominated by the MTTKRP computation.  
Without using a dimension tree, we observe that the KRP is expensive and yields a $2.5\times$ slower total run time even in the 4D case. 

\begin{figure}
\begin{tikzpicture}
\renewcommand{\datafile}{data/wk_4D_syn.dat}
\renewcommand{\numiterations}{10}
\breakdownplot
%\labels
\end{tikzpicture}
\caption{Weak scaling of 4D synthetic tensors with (D) and without (N) the use of dimension trees.  The tensor and processor grid dimensions are $128k\times 128k\times 128k\times 128k$ and $k\times k\times k\times k$ for $k\in\{1,2,3,4\}$, and the rank is fixed at 32.  The reported times are per iteration.}
\label{fig:weaksynthetic4D}
\end{figure}

\subsection{Varying Processor Grid}

In order to illustrate the effect of processor grid choice on running time, we show in \cref{fig:commsweep10} a time breakdown over various processor grid choices for a 4D problem on 81 processors.
Because the tensor is cubical and 81 has a restricted factorization into 4 numbers, there are 5 distinct processor grids.
The overall takeaway is that the processor grid has very little effect on running time; in this experiment there is less than 10\% variation in overall time.
While the optimal processor grid reduced the communication time by approximately $3\times$ compared to the other processor grids, the running time is dominated by local computation, so it had little effect on overall time.
Furthermore, adjusting the processor grid affects the local tensor dimensions and the performance of the local computations, and the optimal processor grid led to slower local performance.
For $R=10$, all of the local computation is memory bandwidth bound, and we believe the variations in running times to be effects of some temporary quantities fitting into smaller levels of cache. 

\begin{figure}
\renewcommand{\datafile}{data/comm_sweep_scale_pgf_10.dat}
\renewcommand{\numiterations}{10}
\begin{tikzpicture}
\begin{axis}[	
	ybar stacked,
	bar width=8pt,
	width=\columnwidth,
	height =.75\columnwidth,
	%width=9cm, height=3.85cm,
	ylabel={Time (s)}, 
	xlabel={Processor Grid},
	y label style={yshift=-.5cm},
	ymin=0,
	%symbolic x coords={D10,N10,,D20,N20,,D30,N30,,D40,N40,,D50,N50},
	symbolic x coords={81x1x1x1-10, 27x3x1x1-10,9x9x1x1-10,9x3x3x1-10,3x3x3x3-10,},
	%xtick=data,
	xticklabels={,81x1x1x1,27x3x1x1,9x9x1x1,9x3x3x1,3x3x3x3},
	legend style={at={(0.5,1.3)},anchor=north},
	legend columns=-1,
]
	\setcolors
	\addplot table[x=algo-k, y expr=(\thisrow{mttkrp}/(\minvalue*\numiterations))] {\datafile};
	\addplot table[x=algo-k, y expr=(\thisrow{mttv}/(\minvalue*\numiterations))] {\datafile};
	\addplot table[x=algo-k, y expr=((\thisrow{nnls}+\thisrow{gram})/(\minvalue*\numiterations))] {\datafile};
	\addplot table[x=algo-k, y expr=((\thisrow{allgather}+\thisrow{reducescatter})/(\minvalue*\numiterations))] {\datafile};
	\addplot table[x=algo-k, y expr=(\thisrow{allreduce}/(\minvalue*\numiterations))] {\datafile};
	\legend{PM,mTTV,NLS,Factor Comm,Gram Comm};
\end{axis}
%\labels
\end{tikzpicture}
\caption{Time breakdown for $243\times243\times243\times243$ tensor and rank $R=10$ on 81 processors for varying processor grids.}
\label{fig:commsweep10}
\end{figure}

\subsection{Varying Approximation Rank}

One of the challenges of the CP (and NNCP) decomposition in practice is the choice of decomposition rank.
The most common technique is to compute multiple CP decompositions for various ranks.
As the rank $R$ increases, the approximation error  $\|\TA - \T{M}\|$ decreases with the better approximation power of more parameters. 
However, the benefit of increasing $R$ eventually diminishes if the data can be well approximated with a CP model.
%that an increase in $R$ in lower values, say from 5 to 10, will have significant improvement in relative error over increase in higher 
%values, like 100 to 105. 
%Hence, it is common practice in the community to sweep $k$, to obtain better approximation error within the 
%manageable computation. 
Towards this end, we experiment with various values of $R$ to observe the relative increase in running time for two real-world data sets. 

\Cref{fig:ksweep4DHSI} shows the time breakdown of our implementation using a dimension tree on the 4D HSI dataset for $R=\{10,\dots,50\}$. 
We observe an overall time increase with increased $R$, but each part of the computation scales slightly differently.
The multi-TTV computation ({\em mTTV}) scales linearly with the increasing $R$,  whereas the partial MTTKRP ({\em PM}) is scaling super-linearly. 
This is because mTTV is cast as matrix-vector products (DGEMV) and PM is cast as matrix-matrix products (DGEMM).
As $R$ increases from 10 to 50, DGEMM performance improves but DGEMV performance is constant.  
The local NLS time is increasing with $O(R^3)$ as expected and the All-Reduce required for the Gram matrices scales with $O(R^2)$, becoming a significant cost for larger $R$. 

In \cref{fig:ksweepneuro} we compare performance for various ranks $R$ across all 3 algorithms, again used flat MPI. 
Starting at $R=10$ we see the largest speed up of $2\times$ for our implementation with a dimension tree over NbAO-NTF. 
This is due to a combination of the dimension tree performing fewer flops in the MTTKRPs and KRPs. 
However, as the rank increases this speed up diminishes to $1.6\times$. 
The loss of speed up is a result of the fact that, as we observed in \cref{fig:ksweep4DHSI}, the multi-TTV operations do not scale as well as the partial-MTTKRPs for increasing $R$.
Again, the performance of our implementation without the dimension tree optimization is comparable to NbAO-NTF. 

\begin{figure}
\renewcommand{\datafile}{data/ksw_4D_HSI.dat}
\renewcommand{\numiterations}{10}
\begin{tikzpicture}
\begin{axis}[	
	ybar stacked,
	bar width=8pt,
	width=\columnwidth,
	height =.75\columnwidth,
	%width=9cm, height=3.85cm,
	ylabel={Time (s)}, 
	xlabel={Rank $R$},
	y label style={yshift=-.5cm},
	ymin=0,
	%symbolic x coords={D10,N10,,D20,N20,,D30,N30,,D40,N40,,D50,N50},
	symbolic x coords={D10,D20,D30,D40,D50},
	%xtick=data,
	xticklabels={,10,20,30,40,50},
	legend style={at={(0.5,1.3)},anchor=north},
	legend columns=-1,
]
	\setcolors
	\addplot table[x=alg-K, y expr=(\thisrow{mttkrp}/(\minvalue*\numiterations))] {\datafile};
	\addplot table[x=alg-K, y expr=(\thisrow{mttv}/(\minvalue*\numiterations))] {\datafile};
	\addplot table[x=alg-K, y expr=((\thisrow{nnls}+\thisrow{gram})/(\minvalue*\numiterations))] {\datafile};
	\addplot table[x=alg-K, y expr=((\thisrow{allgather}+\thisrow{reducescatter})/(\minvalue*\numiterations))] {\datafile};
	\addplot table[x=alg-K, y expr=(\thisrow{allreduce}/(\minvalue*\numiterations))] {\datafile};
	\legend{PM,mTTV,NLS,Factor Comm,Gram Comm};
\end{axis}
%\labels
\end{tikzpicture}
\caption{Per-iteration time breakdown of our implementation (using dimension trees) over various ranks for a time-lapse HSI dataset with dimensions $1344\times 1024\times 33 \times 9$ on 64 processors arranged in a $8\times8\times1\times1$ grid.}
\label{fig:ksweep4DHSI}
\end{figure}

\begin{figure}
\begin{tikzpicture}
\begin{axis}[legend style={draw=none, anchor=north,  cells={align=left}, nodes={scale=0.7}}, legend pos=north west, xlabel=Low Rank $R$, ylabel=Time(s), ylabel near ticks]
	\addplot+ [discard if not={alg}{D},thick,mark options={solid},mark=square*] table [x={k}, y expr=(\thisrow{total}/(\minvalue*\numiterations))] {data/kswp_neuro.dat};
	\addplot+ [discard if not={alg}{N},thick,mark options={solid},mark=square*] table [x={k}, y expr=(\thisrow{total}/(\minvalue*\numiterations))] {data/kswp_neuro.dat};
	\addplot+ [discard if not={alg}{L},thick,mark options={solid},mark=square*] table [x={k}, y expr=(\thisrow{total}/(\minvalue*\numiterations))] {data/kswp_neuro.dat};
	\legend{DimTree,NoDimTree,NbAO-NTF}
\end{axis}
\end{tikzpicture}
\caption{Overall running time for dFC dataset with dimensions $30{,}012\times 1200 \times 500$ on 1440 cores arranged in a $2\times 6\times 120$ processor grid and varying choices of rank $R$.}
\label{fig:ksweepneuro}
\end{figure}


%\begin{figure}

% set which run to plot
%\renewcommand{\run}{3}
%
%\begin{subfigure}{0.3 \columnwidth}
%\ylabeltrue
%\begin{tikzpicture}
%\renewcommand{\datafile}{data/denserwerr-time.dat}
%\renewcommand{\run}{1}
%\relerrplot
%\renewcommand{\run}{3}
%\end{tikzpicture}
%\ylabelfalse
%\subcaption{Video}
%\label{fig:denserwerr}
%\end{subfigure}
%~
%\begin{subfigure}{0.3 \columnwidth}
%\begin{tikzpicture}
%\renewcommand{\datafile}{data/stkx-5runs-err.dat}
%\legendtrue
%\relerrplot
%\legendfalse
%\end{tikzpicture}
%\subcaption{Stack Exchange}
%\label{fig:stackexchangeerr}
%\end{subfigure}
%~
%\begin{subfigure}{0.3 \columnwidth}
%\begin{tikzpicture}
%\renewcommand{\datafile}{data/webbase1M-5runs-err.dat}
%\relerrplot
%\end{tikzpicture}
%\subcaption{Webbase}
%\label{fig:sparserwerr}
%\end{subfigure}
%
%\caption{Relative error comparison of \MU, \HALS, \BPP\ on real world datasets.}
%\label{fig:convergence}
%\end{figure}

% set which run to plot
%\renewcommand{\run}{3}
%
%\begin{subfigure}{0.3 \columnwidth}
%\ylabeltrue
%\begin{tikzpicture}
%\renewcommand{\datafile}{data/denserwerr-time.dat}
%\renewcommand{\run}{1}
%\strongscalingplot
%\renewcommand{\run}{3}
%\end{tikzpicture}
%\ylabelfalse
%\subcaption{Video}
%\label{fig:denserwerr}
%\end{subfigure}
%~
%\begin{subfigure}{0.3 \columnwidth}
%\begin{tikzpicture}
%\renewcommand{\datafile}{data/stkx-5runs-err.dat}
%\legendtrue
%\strongscalingplot
%\legendfalse
%\end{tikzpicture}
%\subcaption{Stack Exchange}
%\label{fig:stackexchangeerr}
%\end{subfigure}
%~
%\begin{subfigure}{0.3 \columnwidth}
%\begin{tikzpicture}
%\renewcommand{\datafile}{data/webbase1M-5runs-err.dat}
%\strongscalingplot
%\end{tikzpicture}
%\subcaption{Webbase}
%\label{fig:sparserwerr}
%\end{subfigure}
%
%\caption{Relative error comparison of \MU, \HALS, \BPP\ on real world datasets.}
%\label{fig:convergence}
%\end{figure}