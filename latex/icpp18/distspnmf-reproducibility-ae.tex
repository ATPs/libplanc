% LaTeX template for Artifact Description appendix used for SC17
% V20170220
%% V20170220
% (C)opyright 2017

% Derived with permission by Michael Heroux (Sandia National Laboratories, St. John's University, MN)
% from ae-20160509.tex 
% written by Grigori Fursin (cTuning foundation, France and dividiti, UK) 
% and Bruce Childers (University of Pittsburgh, USA)
% (C)opyright 2014-2016

% sigplanconf.cls is available at http://www.sigplan.org/sites/default/files/sigplanconf.cls
\documentclass[sigconf, review=false]{acmart}

\usepackage{url}

\newcommand{\distspnmffull}{Distributed sparse NMF algorithm}

\begin{document}

\special{papersize=8.5in,11in}

%%%%%%%%%%%%%%%%%%%%%%%%%%%%%%%%%%%%%%%%%%%%%%%%%%%%
% When adding this appendix to your paper, 
% please remove above part
%%%%%%%%%%%%%%%%%%%%%%%%%%%%%%%%%%%%%%%%%%%%%%%%%%%%



\appendix
\section{Artifact Description: \distspnmffull}

Submission and reviewing guidelines and methodology: \\
{\small\em http://sc17.supercomputing.org/submitters/technical-papers/reproducibility-initiatives-for-technical-papers/artifact-description-paper-title/}

%%%%%%%%%%%%%%%%%%%%%%%%%%%%%%%%%%%%%%%%%%%%%%%%%%%%%%%%%%%%%%%%%%%%%
\subsection{Abstract}

This artifact contains all the components and instruction for the paper titled \distspnmffull\ submitted to SC`2017. 

%%%%%%%%%%%%%%%%%%%%%%%%%%%%%%%%%%%%%%%%%%%%%%%%%%%%%%%%%%%%%%%%%%%%%
\subsection{Description}

\subsubsection{Check-list (artifact meta information)}

{\small
\begin{itemize}
  \item {\bf Algorithm: } \distspnmffull for Multiplicative Update(MU), Hierarchical Alternating Least Squares(HALS) and ANLS/BPP supporting multi constrained hypergraph partitions 
  \item {\bf Program: } C/C++ code
  \item {\bf Compilation: } C/C++ compiler with C++11 support. We tested with GCC 5.3.0 compiler. 
  \item {\bf Transformations: }
  \item {\bf Binary: } distnmf
  \item {\bf Data set: } Flickr and Delicious dataset. Publicly available. 
  \item {\bf Run-time environment: } MPI on Unix
  \item {\bf Hardware: } Rhea on Oak Ridge Leadership Computing Platform and IBM BlueGene/Q supercomputer
  \item {\bf Run-time state: } 
  \item {\bf Execution: } Rhea execution - mpirun -np 16 ./distnmf \-a [0/1/2 for mu/hals/bpp] \-\-pr=[number of row processes] \-\-pc=[number of col processes] \-k [low rank value] \-i [input file] \-t [number of iterations] \\
BGQ execution -  runjob \-\-ranks-per-node 16 \-\-np 16 : /linkhome/rech/qyy/rqyy003/nmflibrary/mpi/distnmf \-a 0 \-i /linkhome/rech/qyy/rqyy003/job-run/flickr.mtx -t 10 -k 16 \\
  \item {\bf Output: }
total\_allgather::m::17261568::n::2473984::k::48::SIZE::1536::algo::0::root::13.66::min::13.36::avg::13.656::max::13.79 \\
total\_allreduce::m::17261568::n::2473984::k::48::SIZE::1536::algo::0::root::0.1::min::0.03::avg::0.106354::max::0.21 \\
total\_reducescatter::m::17261568::n::2473984::k::48::SIZE::1536::algo::0::root::21.99::min::21.8::avg::22.0041::max::22.18 \\
total\_gram::m::17261568::n::2473984::k::48::SIZE::1536::algo::0::root::1.48::min::1.37::avg::1.47555::max::1.64 \\
total\_mm::m::17261568::n::2473984::k::48::SIZE::1536::algo::0::root::10.04::min::9.96::avg::10.0897::max::10.23 \\
total\_nnls::m::17261568::n::2473984::k::48::SIZE::1536::algo::0::root::7.95::min::7.64::avg::7.93462::max::8.05 \\
  \item {\bf Experiment workflow: }
  \begin{enumerate}
\item   git clone source code
\item download dependencies
\item build source code
\item download datasets
\item prepare data
\item create scheduler scripts
\item queue the job
\item analyze the output.
\end{enumerate}
  \item {\bf Experiment customization: } Yes, the software is general
purpose.
  \item {\bf Publicly available?: } Yes.
\end{itemize}
}

\subsubsection{How software can be obtained (if available):} 
Github/Gitlab repository. The clone URL can reveal the person's identity and can violate the double blind review process. 


\subsubsection{Hardware dependencies}. \\

We have tested on two different supercomputers Rhea on Oak Ridge Leadership Computing Facility and IBM Bluegene/Q supercomputer.  

\subsubsection{Software dependencies} \label{sec:dependencies}

\begin{itemize}
\item GCC 5.3.0 and above with support for C++11
\item CMake
\item BLAS/LAPACK libraries - Tested w/ MKL 16.0.0, OpenBLAS (https://github.com/xianyi/OpenBLAS), libblas libraries. 
\item Boost 1.58.0  and above (Needed only for ANLS/BPP algorithm)
\item Armadillo - http://arma.sourceforge.net/
\item MPI Library - tested w/ OpenMPI 1.8.4 and Bluegene/Q MPI library
\end{itemize}


\subsubsection{Datasets} \label{sec:datasets} 

Delicious - \url{https://www.dropbox.com/s/3602tin4g59n8vl/delicious.mtx.bz2?dl=0}
Flickr - \url{https://www.dropbox.com/s/r9xn19vhq0r0dvi/flickr.tar.bz2?dl=0}

%%%%%%%%%%%%%%%%%%%%%%%%%%%%%%%%%%%%%%%%%%%%%%%%%%%%%%%%%%%%%%%%%%%%%
\subsection{Installation} 

Follow the build instruction in README.md

%%%%%%%%%%%%%%%%%%%%%%%%%%%%%%%%%%%%%%%%%%%%%%%%%%%%%%%%%%%%%%%%%%%%%
\subsection{Experiment workflow}

\begin{enumerate}
\item git clone source code (DistSPNMF, pacoss, tmpi)
\item download dependencies listed in Section \ref{sec:dependencies}
\item build source code following the README.md file
\item download datasets listed in Section \ref{sec:datasets}
\item prepare data - {\em pacoss partition} and {\em pacoss distribute}
\item create scheduler scripts - PBS/Torque scripts
\item queue the job 
\item analyze the output.
\end{enumerate}

%%%%%%%%%%%%%%%%%%%%%%%%%%%%%%%%%%%%%%%%%%%%%%%%%%%%%%%%%%%%%%%%%%%%%
\subsection{Evaluation and expected result} 

As explained in the paper experiment section, the $P2PHP$ and $P2PRP$ will provide better running time over $FAUN$ in higher number of processors. The output time can be parsed as explained in output above. 

%%%%%%%%%%%%%%%%%%%%%%%%%%%%%%%%%%%%%%%%%%%%%%%%%%%%%%%%%%%%%%%%%%%%%
\subsection{Experiment customization} 

The software is implemented by extending the $MPI-FAUN$ library and the distribution strategies are available though $pacoss$ software. Users can place with the options available in these software to trying different NMF algorithms and different partition methods. 

%%%%%%%%%%%%%%%%%%%%%%%%%%%%%%%%%%%%%%%%%%%%%%%%%%%%%%%%%%%%%%%%%%%%%
\subsection{Notes} 

The entire software will be available for users through github. 

%%%%%%%%%%%%%%%%%%%%%%%%%%%%%%%%%%%%%%%%%%%%%%%%%%%%
% When adding this appendix to your paper, 
% please remove below part
%%%%%%%%%%%%%%%%%%%%%%%%%%%%%%%%%%%%%%%%%%%%%%%%%%%%

\end{document}
