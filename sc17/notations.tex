\subsection{Notation}
\label{sec:notations}

\Cref{tab:notation} summarizes the notation we use throughout this paper.
We use bold uppercase letters for matrices and lowercase letters for vectors.
For matrix rows and columns, we employ MATLAB notation, i.e., $\AA(i, :)$ and $\AA(:, j)$ refer to the $i$th row and the $j$th column of $\AA$.
We use subscripts to refer to sub-blocks of matrices.
For example, $\AA_ij$ refers to the sub-block $(i, j)$ of $\AA$ in a 2D partition.
We use $m$ and $n$ to denote the numbers of rows and columns of $\AA$, respectively, and assume without loss of generality $m\geq n$ throughout.

\begin{table}%[htdp]
\begin{center}
\begin{tabular}{|l|l|}
\hline
$\AA$ & Input matrix \\
$\WW$ & Left low rank factor \\
$\HH$ & Right low rank factor \\
$m$ & Number of rows of input matrix \\
$n$ & Number of columns of input matrix \\
$k$ & Low rank \\
$P$ & Number of parallel processes \\
$P_r$ & Number of rows in processor grid \\
$P_c$ & Number of columns in processor grid \\
$\Ip$,$\Jp$ & Set of rows/columns of of $\Wm$/$\Hm$ owned by process $p$  \\
$\Fp$,$\Gp$ & Set of unique row and column indices of $\Amp$ \\
$\Amp$ & Submatrix of $\Am$ owned by process p \\
$\Wm(\Ip,:)$ & Owned rows of initial $\Wm$ by process p \\
$\Hm(:,\Jp)$ & Owned columns of initial $\Hm$ by process p\\

\hline
\end{tabular}
\end{center}
\caption{Notation}
\label{tab:notation}
\end{table}%
